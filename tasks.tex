% ---------------------------- Problem 1----------------------------------
\subsubsection*{\center Задача № 1.}
{\bf Условие.~}
Разложить в ряд Фурье заданную функцию $f(x)$, построить графики $f(x)$ и суммы ее ряда Фурье. Если не указывается, какой вид разложения в ряд необходимо представить, то требуетчя разложить функцию либо в общий тригонометрический ряд Фурье, либо следует выбрать оптимальный вид разложения в зависимости от данной функции.
\[
f(x)=x^{3/2}\;  \text{на отрезке}\ \:[0;\,\pi].	
\]

{\bf Решение.~}	
%График
\begin{center}
	\begin{tikzpicture}
	\begin{axis}[xmin=-1,	xmax=3.5, 	ymin=-1,	ymax=6.9,
	width=0.5\textwidth,
	height=0.4\textwidth,
	axis x line=middle,
	axis y line=middle, 
	every axis x label/.style={at={(current axis.right of origin)},anchor=west},
	every inner x axis line/.append style={|-latex'},
	every inner y axis line/.append style={|-latex'},
	minor tick num=1,			
	axis equal=true,
	xlabel=$x$, 
	ylabel=$y$,          
	samples=100,
	clip=true,
	]
	\addplot[color=black, line width=1.5pt,domain=0:pi]{\x^(3/2)};
	\addplot[thick,dashed] coordinates {(pi,0) (pi,pi^(3/2))};
	\addplot[
	mark=*,
	mark options={fill=black, draw=black},
	only marks,
	] coordinates {(pi,pi^(3/2))};
	\end{axis}
	\end{tikzpicture}
\end{center}
\noindent
Доопределим функцию четным образом на отрезке [-\pi;0]:
%График
\begin{center}
	\begin{tikzpicture}
	\begin{axis}[xmin=-1,	xmax=3.5, 	ymin=-1,	ymax=6.9,
	width=0.5\textwidth,
	height=0.4\textwidth,
	axis x line=middle,
	axis y line=middle, 
	every axis x label/.style={at={(current axis.right of origin)},anchor=west},
	every inner x axis line/.append style={|-latex'},
	every inner y axis line/.append style={|-latex'},
	minor tick num=1,			
	axis equal=true,
	xlabel=$x$, 
	ylabel=$y$,          
	samples=100,
	clip=true,
	]
	\addplot[color=black, line width=1.5pt,domain=0:pi]{\x^(3/2)};
	\addplot[color=black, line width=1.5pt,domain=-pi:0]{(-\x)^(3/2)};
	\addplot[thick,dashed] coordinates {(pi,0) (pi,pi^(3/2))};
	\addplot[thick,dashed] coordinates {(-pi,0) (-pi,pi^(3/2))};
	\addplot[
	mark=*,
	mark options={fill=black, draw=black},
	only marks,
	] coordinates {(pi,pi^(3/2))};
	\addplot[
	mark=*,
	mark options={fill=black, draw=black},
	only marks,
	] coordinates {(-pi,pi^(3/2))};
	\end{axis}
	\end{tikzpicture}
\end{center}

\noindent
Для полученной четной функции построим ряд Фурье на отрезке [-\pi;\pi]:


$$
f_1(x)=\frac{a_0}{2}+\sum_{n=1}^\infty 
	\left(a_n\cos{\frac{\pi n x}{L}}\right),\quad\text{где }\,L=\pi
$$
\noindent
Вычислим коэффициенты
$$
\begin{array}{rcl}
a_0 &=& \displaystyle\frac{2}{\pi}\left.
\int\limits_0^\pi
x^\frac{3}{2}\,dx  = 
\frac{2}{\pi}\left(\frac{2x^\frac{5}{2}}{5}
\right)
\right|_0^\pi  = \frac{4\pi^\frac{3}{2}}{5},						\\[12pt]
a_n &=& \displaystyle\frac{2}{\pi}
	\int\limits_0^\pi
	x^\frac{3}{2}\cos {nx}\,dx   = \int_0^\pi x^{\frac{3}{2}} \cos{n x} dx = \frac{\sqrt{\pi}\left(-3 \sqrt{2}\sqrt{n}\;  C(\sqrt{2} \sqrt{n}) + 4 \pi n^2 \sin{\pi n} + 6 n \cos{\pi n}\right)}{4 n^3}.       \\[12pt]
	
\end{array}
$$
\noindent
[C(t) - интеграл Френеля]\\
\noindent
[Ответ взят из пакета WolframAlpha, так как вычислить интеграл известными способами не получилось]\\
\noindent
Применив теорему Дирихле о поточечной сходимости ряда Фурье, видим, что построенный ряд Фурье сходится 
к периодическому (с периодом $T=2\pi$) продолжению функции $f_1(x)$ при всех $x$, так как полученная функция $f_1(x)$ --- четная. 
График функции $S(x)$ имеет следующий вид:

\begin{center}
\begin{tikzpicture}
\begin{axis}[xmin=-7, xmax=7, ymin=-1, ymax=7,
width=0.8\textwidth,
height=0.4\textwidth,
axis x line=middle,
axis y line=middle,
every axis x label/.style={at={(current axis.right of origin)},anchor=west},
every inner x axis line/.append style={|-latex'},
every inner y axis line/.append style={|-latex'},
minor tick num=1,
axis equal=true,
xlabel=$x$,
ylabel=$S(x)$,
samples=100,
clip=true,
]
\addplot[color=black, line width=1.5pt,domain=0:pi]{\x^(3/2)};
	\addplot[color=black, line width=1.5pt,domain=-pi:0]{(-\x)^(3/2)};
	\addplot[thick,dashed] coordinates {(pi,0) (pi,pi^(3/2))};
	\addplot[thick,dashed] coordinates {(-pi,0) (-pi,pi^(3/2))};
	%\addplot[color=black, line width=1.5pt,domain=pi:(2pi)]{(-(\x-2\pi))^(3/2)};
	%\addplot[color=black, line width=1.5pt,domain=(-2pi):-pi]{(\x+2\pi)^(3/2)};
	\addplot[
	mark=*,
	mark options={fill=black, draw=black},
	only marks,
	] coordinates {(-pi,pi^(3/2))};
	\addplot[
	mark=*,
	mark options={fill=black, draw=black},
	only marks,
	] coordinates {(pi,pi^(3/2))};



\end{axis}
\end{tikzpicture}
\end{center}
\noindent
\textbf{Ответ:}
\[
\begin{split}
&f(x)=\frac{4\pi^\frac{3}{2}}{5}+\sum_{n=1}^\infty\left[\frac{\sqrt{\pi}\left(-3 \sqrt{2}\sqrt{n}\;  C(\sqrt{2} \sqrt{n}) + 4 \pi n^2 \sin{\pi n} + 6 n \cos{\pi n}\right)}{4 n^3}\cos{ nx}\right], 
\end{split}
\]




% ---------------------------- Problem 2----------------------------------
\subsubsection*{\center Задача № 2.}
{\bf Условие.~}
Для заданной графически функции $y(x)$ построить ряд Фурье в комплексной форме, изобразить график суммы построенного ряда

%График
\begin{center}
\begin{tikzpicture}[
		declare function={
			func(\x)=
			and(\x >= 0, \x <= 2) * 0.5 * (2 - \x) + 
			(\x > 2) * 0.0;
		}
		]
		\begin{axis}[
		axis x line=middle, axis y line=middle,
		axis equal,	
		ymin=-1.1, ymax=1.1, ytick={-1,...,1}, ylabel=$y$,
		xmin=-1.1, xmax=7, xtick={-1,...,6}, xlabel=$x$,
		domain=0.0:6,samples=600 % added		
		]
		
		\addplot [domain=0:2,black,line width=2pt] {0.5 * (2 - \x)};
		\addplot [domain=2:6,black,line width=2pt] {0.0};		
		\end{axis}
		\end{tikzpicture}
\end{center}

\noindent
\textbf{Решение.}\\

\noindent
Ряд Фурье в комплексной форме имеет следующий вид
\[
f(x) = \sum_{n=-\infty}^\infty c_n e^{i\omega nx},\quad c_n=\frac{1}{T}\int\limits_a^b f(x) e^{-i\omega nx}dx,\,\omega=\frac{2\pi}{T}.
\]
В нашем примере $ a=0,b=6,T=6,\omega=\pi/3$, 
найдем коэффицинеты $c_n,\,n=0,\pm1,\pm2,\ldots$
где $\omega=2\pi/T,\,T=6.$

$$
\begin{array}{rcl}
c_0 &=&\displaystyle\frac{1}{6} \int\limits_0^6 f(x)dx=\frac{a_0}{2}=\frac{1}{6}\left(2-\frac{1}{2}\int\limits_0^2 x dx\right)=\frac{1}{6},\\[12pt]
c_n &=&\displaystyle\frac{1}{6}
\int\limits_0^2
(1-\frac{x}{2})e^{-i\omega nx}dx={}\\[12pt]
&=&\displaystyle\frac{1}{6}\left(\int\limits_0^2 e^{-i\omega nx}dx - \left\frac{1}{2}\int\limits_0^2 xe^{-i\omega nx}dx\right)\right\right=\frac{1}{6}\left(\left\frac{i}{\omega n}e^{-i\omega nx}\right|_0^2-\frac{1}{2}\left(\left\frac{i}{\omega n}e^{-i\omega nx}x\right|_0^2-\frac{i}{\omega n}\int\limits_0^2 e^{-i\omega nx}dx\right)\right)={}\\[12pt]
&=&\displaystyle\frac{1}{6}\left(\frac{i}{\omega n}e^{-2i\omega n}-\frac{i}{\omega n}-\frac{1}{2}\left(\frac{2i}{\omega n}e^{-2i\omega n}-\frac{i}{\omega n}\left(\frac{i}{\omega n}e^{-2i\omega n}-\frac{i}{\omega n}\right)\right)\right)={}\\[12pt]
&=&\displaystyle\frac{1}{6}\left(\frac{i}{\omega n}e^{-2i\omega n}-\frac{i}{\omega n}-\frac{i}{\omega n}e^{-2i\omega n}-\frac{1}{2\omega^2 n^2}e^{-2i\omega n}+\frac{1}{2\omega^2 n^2}\right)={}\\[12pt]
&=&\displaystyle \frac{1}{12\omega^2 n^2}(1-e^{-2i\omega n})-\frac{i}{6\omega n}={}\\[12pt]
&=&\displaystyle \frac{1}{12\omega^2 n^2}(1-cos{\frac{2\pi n}{3}}+isin{\frac{2\pi n}{3}})-\frac{i}{6\omega n}.\end{array}.
$$
\noindent
Применив теорему Дирихле о поточечной сходимости ряда Фурье, видим, что построенный ряд Фурье сходится 
к периодическому (с периодом $T=6$) продолжению исходной функции при всех $x\ne 6n$, и $S(6n)=1/2$ при 
$n=0,\pm1,\pm2,\ldots$, где $S(x)$ --- сумма ряда Фурье. График функции $S(x)$ имеет вид
\begin{center}
	\begin{tikzpicture}
	\begin{axis}[xmin=-6, xmax=6, ymin=-1, ymax=0.5,
	width=0.8\textwidth,
	height=0.4\textwidth,
	axis x line=middle,
	axis y line=middle, 
	every axis x label/.style={at={(current axis.right of origin)},anchor=west},
	every inner x axis line/.append style={|-latex'},
	every inner y axis line/.append style={|-latex'},
	minor tick num=1,			
	axis equal=true,
	xlabel=$x$, 
	ylabel=$S(x)$,          
	samples=100,
	clip=true,
	]
	\addplot[color=black, line width=1.5pt,domain=-6:-4] {-2-\x/2};
	\addplot[color=black, line width=1.5pt,domain=-4:0]{0};
	\addplot[color=black, line width=1.5pt,domain=0:2] {1-\x/2};
	\addplot[color=black, line width=1.5pt,domain=2:6]{0};
	\addplot[thick,dashed] coordinates {(0,0) (0,1)};
	\addplot[
	mark=*,
	mark options={fill=black, draw=black},
	only marks,
	] coordinates {(-6, 0.5)};
		\addplot[
	mark=*,
	mark options={fill=black, draw=black},
	only marks,
	] coordinates {(0, 0.5)};
		\addplot[
	mark=*,
	mark options={fill=black, draw=black},
	only marks,
	] coordinates {(6, 0.5)};
	\end{axis}
	\end{tikzpicture}
\end{center}

\noindent
\textbf{Ответ:}
\[
\begin{split}
&f(x)=\sum_{n=-\infty}^\infty\left[\frac{3}{4\pi^2 n^2}\left(1-\cos{\frac{2\pi n}{3}}+i\sin{\frac{2\pi n}{3}}\right)-\frac{i}{2\pi n} \right] e^{\tfrac{i\pi nx}{3}},~ x\ne 6n; \\
&S(6n)=\frac{1}{2},\quad\text{при}~n\in\mathbb{Z}.
\end{split}
\]

% —------------------------— Problem 3----------------------------------
\subsubsection*{\center Задача № 3.}
{\bf Условие.~}\\
Найти резольвенту для интегрального уравнения Вольтерры со следующим ядром
$$K(x,t)=x^\frac{1}{3}t^\frac{1}{3}.$$

\noindent
{\bf Решение.~}\\
\noindent
Запишем интегральное уравнение Вольтерры
$$
y(x)=x^3+\int\limits_0^x x y(t)dt.
$$
Из рекурентных соотношений получаем
$$
\begin{array}{rcl}
K_1(x,t)&=&\displaystyle x^\frac{1}{3}t^\frac{1}{3}, \\[12pt]
K_2(x,t)&=&\displaystyle\int\limits_t^x K(x,s)K_1(s,t)ds = \int\limits_t^x x^\frac{1}{3}s^\frac{2}{3}t^\frac{1}{3} ds = x^\frac{1}{3}t^\frac{1}{3}\cdot\frac{3 (x^\frac{5}{3}-t^\frac{5}{3})}{5},\\[12pt]
K_3(x,t)&=&\displaystyle\int\limits_t^x K(x,s)K_2(s,t)ds = \int\limits_t^x x^\frac{1}{3}t^\frac{1}{3}\cdot\frac{3 (s^\frac{7}{3}-s^\frac{2}{3}t^\frac{5}{3})}{5} ds = x^\frac{1}{3}t^\frac{1}{3}\frac{3}{5}\int\limits_t^x s^\frac{7}{3}-s^\frac{2}{3}t^\frac{5}{3}ds ={}\\
&=&x^\frac{1}{3}t^\frac{1}{3}\frac{3}{10}(x^{\frac{5}{3}}-t^{\frac{5}{3}})^2.\\[12pt]
K_j(x,t)&=&\displaystylex x^\frac{1}{3}t^\frac{1}{3}\frac{3}{5(j-1)!}\left(x^\frac{5}{3}-t^\frac{5}{3}\right)^{j-1}\!\!\!\!\!\!\!\   j=\mathbb{N}, \text{где}\; K_1(x,t)=x^\frac{1}{3}t^\frac{1}{3}\; \text{задаётся отдельно.}
\end{array}
$$
Подставляя это выражение для итерированных ядер, найдем резольвенту
$$
R(x,t,\lambda)=\frac{3}{5}x^\frac{1}{3}t^\frac{1}{3}\sum_{j=1}^\infty \frac{\lambda^{j-1}}{(j-1)!}\left(x^\frac{5}{3}-t^\frac{5}{3}\right)^{j-1}\!\!\!\!\!\!\!\ ,
\quad j=1,2,\ldots
$$