% ---------------------------- Problem 1----------------------------------
\subsubsection*{\center Задача № 1.}
{\bf Условие.~}
Разложить в ряд Фурье заданную функцию $f(x)$, построить графики $f(x)$ и суммы ее ряда Фурье. Если не указывается, какой вид разложения в ряд необходимо представить, то требуетчя разложить функцию либо в общий тригонометрический ряд Фурье, либо следует выбрать оптимальный вид разложения в зависимости от данной функции.
\[
f(x)=x^{3/2}\;  \text{на отрезке}\ \:[0;\,\pi].	
\]

{\bf Решение.~}	
%График
\begin{center}
	\begin{tikzpicture}
	\begin{axis}[xmin=-1,	xmax=3.5, 	ymin=-1,	ymax=6.9,
	width=0.5\textwidth,
	height=0.4\textwidth,
	axis x line=middle,
	axis y line=middle, 
	every axis x label/.style={at={(current axis.right of origin)},anchor=west},
	every inner x axis line/.append style={|-latex'},
	every inner y axis line/.append style={|-latex'},
	minor tick num=1,			
	axis equal=true,
	xlabel=$x$, 
	ylabel=$y$,          
	samples=100,
	clip=true,
	]
	\addplot[color=black, line width=1.5pt,domain=0:pi]{\x^(3/2)};
	\addplot[thick,dashed] coordinates {(pi,0) (pi,pi^(3/2))};
	\addplot[
	mark=*,
	mark options={fill=black, draw=black},
	only marks,
	] coordinates {(pi,pi^(3/2))};
	\end{axis}
	\end{tikzpicture}
\end{center}
\noindent
Доопределим функцию четным образом на отрезке [-\pi;0]:
%График
\begin{center}
	\begin{tikzpicture}
	\begin{axis}[xmin=-1,	xmax=3.5, 	ymin=-1,	ymax=6.9,
	width=0.5\textwidth,
	height=0.4\textwidth,
	axis x line=middle,
	axis y line=middle, 
	every axis x label/.style={at={(current axis.right of origin)},anchor=west},
	every inner x axis line/.append style={|-latex'},
	every inner y axis line/.append style={|-latex'},
	minor tick num=1,			
	axis equal=true,
	xlabel=$x$, 
	ylabel=$y$,          
	samples=100,
	clip=true,
	]
	\addplot[color=black, line width=1.5pt,domain=0:pi]{\x^(3/2)};
	\addplot[color=black, line width=1.5pt,domain=-pi:0]{(-\x)^(3/2)};
	\addplot[thick,dashed] coordinates {(pi,0) (pi,pi^(3/2))};
	\addplot[thick,dashed] coordinates {(-pi,0) (-pi,pi^(3/2))};
	\addplot[
	mark=*,
	mark options={fill=black, draw=black},
	only marks,
	] coordinates {(pi,pi^(3/2))};
	\addplot[
	mark=*,
	mark options={fill=black, draw=black},
 
%16
	only marks,
	] coordinates {(-pi,pi^(3/2))};
	\end{axis}
	\end{tikzpicture}
\end{center}

\noindent
Для полученной четной функции построим ряд Фурье на отрезке [-\pi;\pi]:


$$
f_1(x)=\frac{a_0}{2}+\sum_{n=1}^\infty 
	\left(a_n\cos{\frac{\pi n x}{L}}\right),\quad\text{где }\,L=\pi
$$
\noindent
Вычислим коэффициенты
$$
\begin{array}{rcl}
a_0 &=& \displaystyle\frac{2}{\pi}\left.
\int\limits_0^\pi
x^\frac{3}{2}\,dx  = 
\frac{2}{\pi}\left(\frac{2x^\frac{5}{2}}{5}
\right)
\right|_0^\pi  = \frac{4\pi^\frac{3}{2}}{5},						\\[12pt]
a_n &=& \displaystyle\frac{2}{\pi}
	\int\limits_0^\pi
	x^\frac{3}{2}\cos {nx}\,dx   =               \\[12pt]
	&=& 
\end{array}
$$
Применив теорему Дирихле о поточечной сходимости ряда Фурье, видим, что построенный ряд Фурье сходится 
к периодическому (с периодом $T=2\pi$) продолжению исходной функции при всех $x\ne 2\pi n$, и 
$S(2\pi n)=\frac{\pi^\frac{3}{2}}{2}$ при $n=0,\pm1,\pm2,\ldots$, где $S(x)$ --- сумма ряда Фурье. 
График функции $S(x)$ имеет следующий вид:
